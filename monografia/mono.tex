\documentclass[12pt,a4paper]{article}
\usepackage[utf8]{inputenc}
\usepackage[T1]{fontenc}
\usepackage{lmodern}
\usepackage{tocloft}
\renewcommand\cftsecleader{\cftdotfill{\cftdotsep}}
\renewcommand{\familydefault}{\sfdefault}
\renewcommand*\contentsname{Sumário}
\usepackage[brazilian]{babel}
\addto\captionsbrazilian{
	\renewcommand{\figurename}{Figura}
	\renewcommand{\tablename}{Tabela}
}
\usepackage{amsmath}
\usepackage{amsfonts}
\usepackage{amssymb}
\usepackage{makeidx}
\usepackage{graphicx}
\usepackage{booktabs}
\usepackage{ragged2e}

\usepackage[left=3.00cm, right=2.00cm, top=3.00cm, bottom=2.00cm]{geometry}
\linespread{1.25}

\author{Ricson Luiz Oliveira Vilaça}
\title{titulo teste teste}

\begin{document}
	
	\centering
	
		\thispagestyle{empty}
		UNIVERSIDADE FEDERAL DE VIÇOSA\\
		CENTRO DE CIÊNCIAS EXATAS E TECNOLÓGICAS\\
		DEPARTAMENTO DE INFORMÁTICA
		\vfill
		{\large \textbf{PROJETO FINAL DE CURSO}}
		\vfill
		{\Large \textbf{TÍTULO}}
		\vfill
		\textbf{Ricson Luiz Oliveira Vilaça} \hfill Graduando em Ciência da Computação
		\vfill
		Michel Melo da Silva\\
		(Orientador)
		\vfill
		VIÇOSA - MINAS GERAIS\\
		2021
	
	\newpage
		\thispagestyle{empty}
		\textbf{Ricson Luiz Oliveira Vilaça}
		\vfill
		{\Large \textbf{TÍTULO}}
		\vfill
		\begin{flushright}
			\begin{minipage}{.5\textwidth}
				Monografia apresentada ao curso de Ciência da Computação da Universidade Federal de Viçosa como parte das exigências para a aprovação na disciplina Seminário I\\\\
				Orientador: Michel Melo da Silva
			\end{minipage}
		\end{flushright}
		\vfill
		VIÇOSA - MINAS GERAIS\\
		2021
	
	\justifying
	
	\newpage
		\tableofcontents
		\thispagestyle{empty}
	
	\newpage
		\centering\section*{RESUMO}\justifying
		
		\noindent\textbf{TÍTULO DO PROJETO}\\
		
		\indent Michel Melo da Silva (Orientador)\\
		\indent Ricson Luiz Oliveira Vilaça (Estudante)\\
		
		\noindent \textbf{RESUMO}\\
		\\\\\\\\\\
		
		\noindent \textbf{PALAVRAS-CHAVE}\\
		\indent Citar 3 palavras-chave\\
		
		\noindent \textbf{ÁREA DE CONHECIMENTO}\\
		\indent Consultar tabela do CNPq e informar o nome e o código.\\
		
		\noindent \textbf{LINHA DE PESQUISA}\\
		\indent Consultar tabela no site do DPI e informar o nome e o código.\\
		
	\newpage
		\section{INTRODUÇÃO}
		\subsection{Objetivos}
		
	\newpage
		\section{REFERENCIAL TEÓRICO}
	
	\newpage
		\section{METODOLOGIA}
		\subsection{Requisitos}
	
	\newpage
		\section{REFERÊNCIAS BIBLIOGRÁFICAS}
		
	\newpage
		\section{ANEXOS}
		
	\newpage
		\section{CRONOGRAMA}
		\begin{table}[htbp]
			\centering
			\caption{Cronograma do Projeto em Meses (2021)}
			\label{tab:cronograma}
			\begin{tabular}{lcccccccccccc} %|c|c|c|c|c|c|c|c|c|c|c|c
				\toprule
				\textbf{Atividade} & \textbf{Fev} & \textbf{Mar} & \textbf{Abr} & \textbf{Mai} & \textbf{Jun} & \textbf{Jul} & \textbf{Ago} & \textbf{Set} & \textbf{Out} & \textbf{Nov} \\
				\midrule
				Estudo de CNNs & $\bullet$\\
				Estudo do dataset & $\bullet$ & $\bullet$\\
				Busca por rede & & $\bullet$\\
				Estudo do framework & & $\bullet$ & $\bullet$ & $\bullet$\\
				Fine tuning & & & $\bullet$ & $\bullet$\\
				Análise do resultado & & & & $\bullet$ & $\bullet$\\
				Modificações na rede & & & & & $\bullet$ & $\bullet$\\
				Modificações no input & & & & & & $\bullet$ & $\bullet$\\
				Analisar resultado & & & & & & & $\bullet$ & $\bullet$\\
				Escrita final & & & & & & & $\bullet$ & $\bullet$ & $\bullet$ & $\bullet$\\
				\bottomrule
			\end{tabular}%
		\end{table}
	

\end{document}