\documentclass[
% -- opções da classe memoir --
12pt,				    % tamanho da fonte
openright,			    % capítulos começam em pág ímpar (insere página vazia caso preciso)
oneside,			    % para impressão só no anverso. Oposto a twoside
a4paper,			    % tamanho do papel.
% -- opções do pacote abntex2 --
% chapter=TITLE,         % Títulos em maiúsculas
sumario=tradicional,    % Sumário padrão memoir (mais bonito "imo")
% -- opções do pacote babel --
english,			    % idioma adicional para hifenização
brazil,				    % o último idioma é o principal do documento
]{abntex2}              % Personaliza a capa. Precisa do arquivo ufv.cls para funcionar.



% Pacotes fundamentais
\usepackage{abntex2-UFV}        % Personalização para a Universidade Federal de Viçosa
\usepackage{lmodern}			% Usa a fonte Latin Modern			
\usepackage[T1]{fontenc}		% Selecao de codigos de fonte de saída
\usepackage[utf8]{inputenc}		% Codificacao do documento (conversão automática dos acentos)
\usepackage{indentfirst}		% Indenta o primeiro parágrafo de cada seção.
\usepackage{graphicx}			% Inclusão de gráficos
\usepackage{booktabs}           % \toprule, \midrule e \bottomrule para tabelas
% Sistema autor-data com títulos nas referências em negrito
\usepackage[alf,abnt-emphasize=bf]{abntex2cite}	


% ---
% CONFIGURAÇÕES DE PACOTES
% ---

% Informações de dados para CAPA e FOLHA DE ROSTO
\titulo{Projeto Final de Curso}
\autor{Ricson Luiz Oliveira Vilaça}
\local{Viçosa}
\data{2021}
\orientador{Michel Melo da Silva}    % redefinido no abntex2-UFV para aceitar Instituição (default = UFV-CRP)
%\coorientador{Nome do Coorientador}
\instituicao{Universidade Federal de Viçosa}

\campus{\emph{Campus} Viçosa}      % pacote abntex2-UFV
\curso{Ciência da Computação}               % pacote abntex2-UFV
%\membrobancaA{Membro da Banca A}             % pacote abntex2-UFV default = UFV-CRP
%\membrobancaB[UFMG]{Membro da Banca B}       % pacote abntex2-UFV default = UFV-CRP
%\databanca{\today}                           % pacote abntex2-UFV

% O preambulo deve conter o tipo do trabalho, o objetivo,
% o nome da instituição e a área de concentração
\preambulo{Monografia apresentada ao curso de Ciência da Computação da Universidade Federal de Viçosa como parte das exigências para a aprovação na disciplina Seminário I}
% ---

% ---
% Configurações de aparência do PDF final

% informações para o arquivo pdf de saída
% Interessante alterar a cor dos links para preto(black)
% para imprimir
\makeatletter
\hypersetup{
	% metadados
	pdftitle={\@title},
	pdfauthor={\@author},
	pdfsubject={\imprimirpreambulo},
	pdfcreator={LaTeX with abnTeX2},
	colorlinks=true,   % false: links em frame; true: links coloridos
	linkcolor=black,    % cor dos links no documento
	citecolor=blue,    % cor dos links para a bibliografia
	filecolor=magenta, % cor dos links para arquivos
	urlcolor=blue,     % cor dos links para sites
	bookmarksdepth=4   % profundidade do sumário do PDF
}
\makeatother
% ---

\begin{document}
	% Retira espaço extra obsoleto entre as frases.
	\frenchspacing
	
	% ----------------------------------------------------------
	% ELEMENTOS PRÉ-TEXTUAIS
	% ----------------------------------------------------------
	\pretextual
	
	% Capa
	\imprimircapa
	
	% Folha de rosto
	\imprimirfolhaderosto
	% ---
	
	% Inserir folha de aprovação
	%\imprimirfolhadeaprovacao
	
	% Dedicatória
	%\begin{dedicatoria}
	%   \vspace*{\fill}
	%   \centering
	%   \noindent
	%   \textit{Texto qualquer da dedicatória}
	%   \vspace*{\fill}
	%\end{dedicatoria}
	% ---
	
	% Agradecimentos
	%\begin{agradecimentos}
	
	%\end{agradecimentos}
	---
	
	% Epígrafe
	%\begin{epigrafe}
	%    \vspace*{\fill}
	%	\begin{flushright}
	%		\textit{``Word? nunca mais.''\\
	%		(Qualquer usuário de \LaTeX)}
	%	\end{flushright}
	%\end{epigrafe}
	% ---
	
	% RESUMOS
	
	% resumo em português
	\begin{resumo}
		\noindent
		%Insira o resumo aqui
		
		\vspace{\onelineskip}
		
		\noindent
		\textbf{Palavras-chave}: Inteligência artifical, visão computacional, redes neurais convolucionais.
	\end{resumo}
	
	% resumo em inglês
	\begin{resumo}[Abstract]
		\begin{otherlanguage*}{english}
			\noindent
			%   % Insira o abstract aqui
			%
			\vspace{\onelineskip}
			
			\noindent
			\textbf{Key-words}: Artificial intelligence, computer vision, convolutional neural networks.
		\end{otherlanguage*}
	\end{resumo}
	
	% inserir lista de ilustrações
	\pdfbookmark[0]{\listfigurename}{lof}
	\listoffigures*
	\cleardoublepage
	% ---
	
	% inserir lista de tabelas
	\pdfbookmark[0]{\listtablename}{lot}
	\listoftables*
	\cleardoublepage
	% ---
	
	
	% Lista de siglas e abreviaturas (opcional)
	% sintaxe: \item [sigla] Descrição da sigla
	
	%\begin{siglas}
	%\item[ABNT] Absurdas Normas Técnicas
	%\item[UFV] Universidade Federal de Viçosa
	%\item[CRP] \emph{Campus} de Rio Paranaíba
	%\end{siglas}
	
	% Lista de símbolos (opcional)
	% sintaxe: \item [simbolo] Descrição do símbolo
	
	%\begin{simbolos}
	%\item[$\infty$ ] Infinito
	%\end{simbolos}
	
	
	% inserir o sumario
	\pdfbookmark[0]{\contentsname}{toc}
	\tableofcontents*
	\cleardoublepage
	% ---
	
	
	
	% ----------------------------------------------------------
	% ELEMENTOS TEXTUAIS
	% ----------------------------------------------------------
	\textual
	
	
	%Modifique a estrutura dos capítulos e seções de acordo com a necessidade do seu trabalho
	\chapter{Introdução}\label{sec:introducao}
	Lorem ipsum dolor sit amet, consectetur adipiscing elit, sed do eiusmod tempor incididunt ut labore et dolore magna aliqua. Ut enim ad minim veniam, quis nostrud exercitation ullamco laboris nisi ut aliquip ex ea commodo consequat. Duis aute irure dolor in reprehenderit in voluptate velit esse cillum dolore eu fugiat nulla pariatur. Excepteur sint occaecat cupidatat non proident, sunt in culpa qui officia deserunt mollit anim id est laborum.
		
	\section{Objetivos}
	Sed ut perspiciatis unde omnis iste natus error sit voluptatem accusantium doloremque laudantium, totam rem aperiam, eaque ipsa quae ab illo inventore veritatis et quasi architecto beatae vitae dicta sunt explicabo. Nemo enim ipsam voluptatem quia voluptas sit aspernatur aut odit aut fugit, sed quia consequuntur magni dolores eos qui ratione voluptatem sequi nesciunt. Neque porro quisquam est, qui dolorem ipsum quia dolor sit amet, consectetur, adipisci velit, sed quia non numquam eius modi tempora incidunt ut labore et dolore magnam aliquam quaerat voluptatem. Ut enim ad minima veniam, quis nostrum exercitationem ullam corporis suscipit laboriosam, nisi ut aliquid ex ea commodi consequatur? Quis autem vel eum iure reprehenderit qui in ea voluptate velit esse quam nihil molestiae consequatur, vel illum qui dolorem eum fugiat quo voluptas nulla pariatur?
	
	\chapter{Referencial Teórico}\label{sec:refTeorico}
	\begin{figure}[htbp]
		\begin{center}
			\includegraphics[width=.5\linewidth]{LogoUFV.png}\\
		\end{center}
		\caption[Exemplo de Figura]{Exemplo de inserção de figura no \LaTeX. A legenda deve vir abaixo da figura. Pode usar o comando \texttt{\textbackslash legend} ou \texttt{\textbackslash fonte} para inserir a fonte da figura. Observe que na lista de ilustrações foi utilizado o nome curto fornecido como parâmetro do caption da figura (veja o arquivo fonte .tex) ao invés dessa legenda estupidamente extensa feita de forma proposital}
		\label{fig:logo}
		\legend{Fonte: Próprio Autor}
	\end{figure}
	
	\chapter{Medodologia}\label{sec:metodos}
	
	Lorem ipsum dolor sit amet, consectetur adipiscing elit, sed do eiusmod tempor incididunt ut labore et dolore magna aliqua. Ut enim ad minim veniam, quis nostrud exercitation ullamco laboris nisi ut aliquip ex ea commodo consequat. Duis aute irure dolor in reprehenderit in voluptate velit esse cillum dolore eu fugiat nulla pariatur. Excepteur sint occaecat cupidatat non proident, sunt in culpa qui officia deserunt mollit anim id est laborum.
	
	\section{Requisitos}
	Sed ut perspiciatis unde omnis iste natus error sit voluptatem accusantium doloremque laudantium, totam rem aperiam, eaque ipsa quae ab illo inventore veritatis et quasi architecto beatae vitae dicta sunt explicabo. Nemo enim ipsam voluptatem quia voluptas sit aspernatur aut odit aut fugit, sed quia consequuntur magni dolores eos qui ratione voluptatem sequi nesciunt. Neque porro quisquam est, qui dolorem ipsum quia dolor sit amet, consectetur, adipisci velit, sed quia non numquam eius modi tempora incidunt ut labore et dolore magnam aliquam quaerat voluptatem. Ut enim ad minima veniam, quis nostrum exercitationem ullam corporis suscipit laboriosam, nisi ut aliquid ex ea commodi consequatur? Quis autem vel eum iure reprehenderit qui in ea voluptate velit esse quam nihil molestiae consequatur, vel illum qui dolorem eum fugiat quo voluptas nulla pariatur?
	
	\chapter{Anexos}\label{sec:anexos}
	
	\chapter{Cronograma}\label{sec:cronograma}
	
	\begin{table}[htbp]
		\centering
		\caption{Cronograma do Projeto em Meses (2021)}
		\label{tab:cronograma}
		\begin{tabular}{lcccccccccccc} %|c|c|c|c|c|c|c|c|c|c|c|c
			\toprule
			\textbf{Atividade} & \textbf{Fev} & \textbf{Mar} & \textbf{Abr} & \textbf{Mai} & \textbf{Jun} & \textbf{Jul} & \textbf{Ago} & \textbf{Set} & \textbf{Out} & \textbf{Nov} \\
			\midrule
			Estudo de CNNs & $\bullet$\\
			Estudo do dataset & $\bullet$ & $\bullet$\\
			Busca por rede & & $\bullet$\\
			Estudo do framework & & $\bullet$ & $\bullet$ & $\bullet$\\
			Fine tuning & & & $\bullet$ & $\bullet$\\
			Análise do resultado & & & & $\bullet$ & $\bullet$\\
			Modificações na rede & & & & & $\bullet$ & $\bullet$\\
			Modificações no input & & & & & & $\bullet$ & $\bullet$\\
			Analisar resultado & & & & & & & $\bullet$ & $\bullet$\\
			Escrita final & & & & & & & $\bullet$ & $\bullet$ & $\bullet$ & $\bullet$\\
			\bottomrule
		\end{tabular}%
	\end{table}
		
	
	% ----------------------------------------------------------
	% ELEMENTOS PÓS-TEXTUAIS
	% ----------------------------------------------------------
	\postextual
	
	% Referências bibliográficas
	
	\bibliography{referencias}
	
	% Caso sejam necessários apêndices ou anexos em seu documento
	% Use os ambientes abaixo
	
	%% Apêndices
	%
	%% Inicia os apêndices
	%\begin{apendicesenv}
	%
	%% Imprime uma página indicando o início dos apêndices
	%\partapendices
	%
	%\chapter{Primeiro Apêndice}
	%
	%\chapter{Segundo Apêndice}
	%
	%\end{apendicesenv}
	%
	%
	%% ----------------------------------------------------------
	%% Anexos
	%% ----------------------------------------------------------
	%\begin{anexosenv}
	%
	%% Imprime uma página indicando o início dos anexos
	%\partanexos
	%
	%\chapter{Primeiro Anexo}
	%\lipsum[30]
	%
	%\chapter{Segundo Anexo}
	%\lipsum[31]
	%
	%\end{anexosenv}
	
\end{document}
